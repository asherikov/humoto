\part{Walking Pattern Generators}\label{part.wpg}
\begin{refsection}


\chapter{Introduction}
\newcommand{\cpos}{\V{c}}
\newcommand{\cvel}{\dotV{c}}
\newcommand{\cacc}{\ddotV{c}}
\newcommand{\cjerk}{\dddotV{c}}

\newcommand{\cPos}{\V{C}}
\newcommand{\cVel}{\dotV{C}}
\newcommand{\cAcc}{\ddotV{C}}
\newcommand{\cJerk}{\dddotV{C}}

\newcommand{\cstate}{\hatV{c}}
\newcommand{\cState}{\hatV{C}}
%\newcommand{\cstate}{\V{\varsigma}}
%\newcommand{\cState}{\V{\zeta}}

\newcommand{\cop}[1][{}]{{\V[#1]{z}}}
\newcommand{\dcop}[1][{}]{{\dotV[#1]{z}}}
%\newcommand{\cop}{\V{z}}
\newcommand{\CoP}{\V{Z}}
\newcommand{\dCoP}{\dotV{Z}}

\newcommand{\fph}{\hatV{p}}
\newcommand{\FPh}{\hatV{P}}

\newcommand{\fp}[1][{}]{{\V[#1]{p}}}
\newcommand{\FP}{\V{P}}

\newcommand{\fd}{\V{f}}
\newcommand{\FD}{\V{F}}

\newcommand{\cp}{\V{\xi}}



%%%%%%%%%%%%%%%%%%%%%%%%%%%%%%%%%%%%%%%%%%%%%%%%%%%%%%%%%%%%%%%%%%%%%%%%%%%%%%%%%%%%%%%%%%%%%%%%%%%%%%
%%%%%%%%%%%%%%%%%%%%%%%%%%%%%%%%%%%%%%%%%%%%%%%%%%%%%%%%%%%%%%%%%%%%%%%%%%%%%%%%%%%%%%%%%%%%%%%%%%%%%%
%%%%%%%%%%%%%%%%%%%%%%%%%%%%%%%%%%%%%%%%%%%%%%%%%%%%%%%%%%%%%%%%%%%%%%%%%%%%%%%%%%%%%%%%%%%%%%%%%%%%%%
\section{Single point-mass models with coplanar contacts}


%%%%%%%%%%%%%%%%%%%%%%%%%%%%%%%%%%%%%%%%%%%%%%%%%%%%%%%%%%%%%%%%%%%%%%%%%%%%%%%%%%%%%%%%%%%%%%%%%%%%%%
\subsection{Common variables}

Position, velocity, acceleration, jerk of the \acs{CoM}:
\begin{align}
\cpos = \begin{bmatrix} c^x \\ c^y \end{bmatrix}
\quad&
\cPos = \begin{bmatrix} \cpos_1 \\ \vdots \\ \cpos_N \end{bmatrix}
\quad\quad\quad&
\cvel = \begin{bmatrix} \dot{c}^x \\ \dot{c}^y \end{bmatrix}
\quad&
\cVel = \begin{bmatrix} \cvel_1 \\ \vdots \\ \cvel_N \end{bmatrix}
\\
%
\cacc = \begin{bmatrix} \ddot{c}^x \\ \ddot{c}^y \end{bmatrix}
\quad&
\cAcc = \begin{bmatrix} \cacc_1 \\ \vdots \\ \cacc_N \end{bmatrix}
\quad\quad\quad&
\cjerk = \begin{bmatrix} \dddot{c}^x \\ \dddot{c}^y \end{bmatrix}
\quad&
\cJerk = \begin{bmatrix} \cjerk_0 \\ \vdots \\ \cjerk_{N-1} \end{bmatrix}
\end{align}


%%%%%%%%%%%%%%%%%%%%%%%%%%%%%%%%%%%%%%%%%%%%%%%%%%%%%%%%%%%%%%%%%%%%%%%%%%%%%%%%%%%%%%%%%%%%%%%%%%%%%%
\subsection{Discrete-time systems based on triple integrator}

The state of the \acs{CoM}:
\begin{equation}
\cstate = \begin{bmatrix} \cstate^x \\ \cstate^y \end{bmatrix} =
          \begin{bmatrix} c^x \\ \dot{c}^x \\ \ddot{c}^x \\ c^y \\ \dot{c}^y \\ \ddot{c}^y \end{bmatrix}
\quad\quad
\cState = \begin{bmatrix} \cstate_1 \\ \vdots \\ \cstate_N \end{bmatrix}\\
\end{equation}


Selection matrices:
\begin{equation}
    \M{I}_{p} =
    \begin{bmatrix}
        1 & 0 & 0 & 0 & 0 & 0 \\
        0 & 0 & 0 & 1 & 0 & 0
    \end{bmatrix}
    \quad
    \M{I}_{v} =
    \begin{bmatrix}
        0 & 1 & 0 & 0 & 0 & 0 \\
        0 & 0 & 0 & 0 & 1 & 0
    \end{bmatrix}
    \quad
    \M{I}_{a} =
    \begin{bmatrix}
        0 & 0 & 1 & 0 & 0 & 0 \\
        0 & 0 & 0 & 0 & 0 & 1
    \end{bmatrix}
\end{equation}
\begin{equation}
    \M{I}_{pv} =
    \begin{bmatrix}
        1 & 0 & 0 & 0 & 0 & 0 \\
        0 & 1 & 0 & 0 & 0 & 0 \\
        0 & 0 & 0 & 1 & 0 & 0 \\
        0 & 0 & 0 & 0 & 1 & 0 \\
    \end{bmatrix}
\end{equation}



%%%%%%%%%%%%%%%%%%%%%%%%%%%%%%%%%%%%%%%%%%%%%%%%%%%%%%%%%%%%%%%%%%%%%%%%%%%%%%%%%%%%%%%%%%%%%%%%%%%%%%
\subsubsection{Piece-wise constant jerk}

The dynamic system:
\begin{equation}
\V{\dot{x}} = \M{A} \V{x} + \M{B} \V{u}
\label{eq: dynamic system}
\end{equation}

can be defined as:
\begin{equation}
\begin{split}
  \V{x} = \cstate \\
  \V{u} = \cjerk
\end{split}
\end{equation}

resulting in:
\begin{equation}
\M{A}_k =
\begin{bmatrix}
    0 & 1 & 0   & 0 & 0 & 0\\
    0 & 0 & 1   & 0 & 0 & 0\\
    0 & 0 & 0   & 0 & 0 & 0\\
    0 & 0 & 0   & 0 & 1 & 0\\
    0 & 0 & 0   & 0 & 0 & 1\\
    0 & 0 & 0   & 0 & 0 & 0\\
\end{bmatrix}
\quad
\M{B}_k =
\begin{bmatrix}
    0         & 0\\
    0         & 0\\
    1  & 0\\
    0         & 0 \\
    0         & 0 \\
    0         & 1       \\
\end{bmatrix}
\end{equation}

After discretization, the model is now:
\begin{equation}
\begin{split}
    \cstate_{k+1} =& \M{A}_k \cstate_k + \M{B}_k \cjerk_k\\
    \cop_{k+1} =& \M{D}_{k} \cstate_{k+1}
\end{split}
\end{equation}

\begin{equation}
\M{A}_k =
\begin{bmatrix}
    1       & T_k   & T_k^2/2   & 0 & 0 & 0\\
    0       & 1     & T_k       & 0 & 0 & 0\\
    0       & 0     & 1         & 0 & 0 & 0\\
    0 & 0 & 0                   & 1       & T_k   & T_k^2/2   \\
    0 & 0 & 0                   & 0       & 1     & T_k       \\
    0 & 0 & 0                   & 0       & 0     & 1         \\
\end{bmatrix}
\quad
\M{B}_k =
\begin{bmatrix}
    T_k^3/6 & 0\\
    T_k^2/2 & 0\\
    T       & 0\\
    0       & T_k^3/6 \\
    0       & T_k^2/2 \\
    0       & T       \\
\end{bmatrix}
\end{equation}

\begin{equation}
\M{D}_{k} =
\begin{bmatrix}
    1       & 0     & -\frac{1}{\omega_k^2} & 0 & 0 & 0\\
    0 & 0 & 0                               & 1       & 0     & -\frac{1}{\omega_k^2}  \\
\end{bmatrix}
\quad
\omega_k = \sqrt{\frac{g}{c^z_k}}
\end{equation}


%%%%%%%%%%%%%%%%%%%%%%%%%%%%%%%%%%%%%%%%%%%%%%%%%%%%%%%%%%%%%%%%%%%%%%%%%%%%%%%%%%%%%%%%%%%%%%%%%%%%%%
\subsubsection{Piece-wise constant CoP velocity}

The same dynamic system of Eq.(\ref{eq: dynamic system}) can be defined as:
\begin{equation}
\begin{split}
  \V{x} = \cstate \\
  \V{u} = \dcop
\end{split}
\end{equation}

resulting in:
\begin{equation}
\M{A}_k =
\begin{bmatrix}
    0 & 1 & 0          & 0 & 0 & 0\\
    0 & 0 & 1          & 0 & 0 & 0\\
    0 & \omega^2 & 0  & 0 & 0 & 0\\
    0 & 0 & 0          & 0 & 1 & 0\\
    0 & 0 & 0          & 0 & 0 & 1\\
    0 & 0 & 0          & 0 & \omega^2 & 0\\
\end{bmatrix}
\quad
\M{B}_k =
\begin{bmatrix}
    0         & 0\\
    0         & 0\\
    -\omega^2 & 0\\
    0         & 0 \\
    0         & 0 \\
    0         & -\omega^2       \\
\end{bmatrix}
\end{equation}

Discretization in Maxima (see \cite{wiki2017discretization}):
\begin{listingtcb}{Maxima}
\begin{deflisting}
load("diag");
A: matrix([0, 1, 0], [0, 0, 1], [0, w^2, 0]);
B: matrix([0],[0],[-w^2]);
Ad: mat_function(exp, A*T);
Bd: expand(integrate(mat_function(exp, A*t), t, 0, T).B);
\end{deflisting}
\end{listingtcb}

The resulting model is:
\begin{equation}
\begin{split}
    \cstate_{k+1} =& \M{A}_k \cstate_k + \M{B}_k \dcop_k\\
    \cop_{k+1} =& \M{D}_{k} \cstate_{k+1}
\end{split}
\end{equation}

\begin{equation}
    \M{A}_k =
    \begin{bmatrix}
        1       & \frac{\sinh(T_k\omega_k)}{\omega_k} & \frac{\cosh(T_k\omega_k) - 1}{\omega_k^2} & 0 & 0 & 0\\
        0       & \cosh(T_k\omega_k)                & \frac{\sinh(T_k\omega_k)}{\omega_k}       & 0 & 0 & 0\\
        0       & \omega_k\sinh(T_k\omega_k)          & \cosh(T_k\omega_k)                      & 0 & 0 & 0\\
        0 & 0 & 0                   & 1       & \frac{\sinh(T_k\omega_k)}{\omega_k} & \frac{\cosh(T_k\omega_k) - 1}{\omega_k^2} \\
        0 & 0 & 0                   & 0       & \cosh(T_k\omega_k)                & \frac{\sinh(T_k\omega_k)}{\omega_k}       \\
        0 & 0 & 0                   & 0       & \omega_k\sinh(T_k\omega_k)          & \cosh(T_k\omega_k)                      \\
    \end{bmatrix}
\end{equation}
\begin{equation}
    \M{B}_k =
    \begin{bmatrix}
        -\frac{\sinh(T_k\omega_k)}{\omega_k} + T_k  & 0\\
        -\cosh(T_k\omega_k) + 1                     & 0\\
        -\omega \sinh(T_k\omega_k)                  & 0\\
        0       & -\frac{\sinh(T_k\omega_k)}{\omega_k} + T_k \\
        0       & -\cosh(T_k\omega_k) + 1                    \\
        0       & -\omega \sinh(T_k\omega_k)                 \\
    \end{bmatrix}
\end{equation}

The state of the system and the output matrix are the same as above.

%%%%%%%%%%%%%%%%%%%%%%%%%%%%%%%%%%%%%%%%%%%%%%%%%%%%%%%%%%%%%%%%%%%%%%%%%%%%%%%%%%%%%%%%%%%%%%%%%%%%%%
\subsubsection{Piece-wise constant CoP velocity (control is the CoP position)}
Let the system controlled by piece-wise constant \acs{CoP} velocity be defined as
%
\begin{equation}
\begin{split}
    \cstate_{k+1} =& \M{\tilde{A}}_k \cstate_k + \M{\tilde{B}}_k \dcop_k\\
    \cop_{k+1} =& \M{\tilde{D}}_{k} \cstate_{k+1},
\end{split}
\end{equation}
%
where matrices $\M{\tilde{A}}_k, \M{\tilde{B}}_k, \M{\tilde{D}}_{k}$ are
defined as above.


We can express $\cop_{k+1}$ using $\cstate_k$ and $\dcop_k$ as
%
\begin{equation}
    \cop_{k+1} = \M{\tilde{D}}_{k} \cstate_k + T_k \dcop_k.
\end{equation}
%
Hence we express the \acs{CoP} velocity through \acs{CoM} and \acs{CoP} positions
%
\begin{equation}
    \dcop_k = \frac{1}{T_k} \left(\cop_{k+1} - \M{\tilde{D}}_{k} \cstate_k \right),
\end{equation}
%
and substitute it back into our model:
%
\begin{equation}
\begin{split}
\cstate_{k+1} =
    \M{\tilde{A}}_k \cstate_k
    +
    \frac{1}{T_k} \M{\tilde{B}}_k \cop_{k+1}
    -
    \frac{1}{T_k} \M{\tilde{B}}_k \M{\tilde{D}}_{k} \cstate_k
    =
    \left( \M{\tilde{A}}_k - \frac{1}{T_k} \M{\tilde{B}}_k \M{\tilde{D}}_{k} \right) \cstate_k
    +
    \frac{1}{T_k} \M{\tilde{B}}_k \cop_{k+1}
\end{split}
\end{equation}
%
Thus we obtain a new model
%
\begin{align}
    \cstate_{k+1} =& \M{A} \cstate_k + \M{B} \cop_{k+1}\\
    \dcop_k =& \M{D}_k \cstate_k + \M{E}_k \cop_{k+1},
\end{align}
%
where
%
\begin{align}
    \M{A}_k &= \left( \M{\tilde{A}}_k - \frac{1}{T_k} \M{\tilde{B}}_k \M{\tilde{D}}_k \right)
    &
    \M{B}_k &= \frac{1}{T_k} \M{\tilde{B}}_k \\
    \M{D}_k &= - \frac{1}{T_k} \M{\tilde{D}}_k
    &
    \M{E}_k &= \diag{2}{\frac{1}{T_k}}
\end{align}
%


\begin{equation}
    \M{A}_k =
    \begin{bmatrix}
        {{\operatorname{sh}}\over{\omega_k\,T_k}}   &   {{\operatorname{sh}}\over{\omega_k}}    &   -{{\operatorname{sh}-\omega_k\,T_k\,\operatorname{ch}}\over{\omega_k^3\,T_k}}   & 0 & 0 & 0\\
        {{\operatorname{ch}-1}\over{T_k}}           &   \operatorname{ch}                       &   {{\omega_k\,T_k\,\operatorname{sh}-\operatorname{ch}+1}\over{\omega_k^2\,T_k}}  & 0 & 0 & 0\\
        {{\omega_k\,\operatorname{sh}}\over{T_k}}   &   \omega_k\,\operatorname{sh}             &   -{{\operatorname{sh}-\omega_k\,T_k\, \operatorname{ch}}\over{\omega_k\,T_k}}     & 0 & 0 & 0\\
        0 & 0 & 0                   & {{\operatorname{sh}}\over{\omega_k\,T_k}}   &   {{\operatorname{sh}}\over{\omega_k}}   &   -{{\operatorname{sh}-\omega_k\,T_k\,\operatorname{ch}}\over{\omega_k^3\,T_k}}\\
        0 & 0 & 0                   & {{\operatorname{ch}-1}\over{T_k}}           &   \operatorname{ch}                       &   {{\omega_k\,T_k\,\operatorname{sh}-\operatorname{ch}+1}\over{\omega_k^2\,T_k}}\\
        0 & 0 & 0                   & {{\omega_k\,\operatorname{sh}}\over{T_k}}   &   \omega_k\,\operatorname{sh}             &   -{{\operatorname{sh}-\omega_k\,T_k\, \operatorname{ch}}\over{\omega_k\,T_k}}\\
    \end{bmatrix},
\end{equation}
%
\begin{equation}
    \M{B}_k
    =
    \begin{bmatrix}
        -{{\operatorname{sh}-\omega_k\,T_k}\over{\omega_k\,T_k}} & 0\\
        -{{\operatorname{ch}-1}\over{T_k}}                       & 0\\
        -{{\omega_k\,\operatorname{sh}}\over{T_k}}               & 0\\
        0       & -{{\operatorname{sh}-\omega_k\,T_k}\over{\omega_k\,T_k}}\\
        0       & -{{\operatorname{ch}-1}\over{T_k}}\\
        0       & -{{\omega_k\,\operatorname{sh}}\over{T_k}}\\
    \end{bmatrix},
\end{equation}
%
where
%
\begin{equation}
    \begin{split}
        \operatorname{sh} &= \sinh (\omega_k\,T_k)\\
        \operatorname{ch} &= \cosh (\omega_k\,T_k)\\
    \end{split}
\end{equation}


Obtaining the matrices in Maxima:

\begin{listingtcb}{Maxima}
\begin{deflisting}
load("diag");
A: matrix([0, 1, 0], [0, 0, 1], [0, w^2, 0]);
B: matrix([0],[0],[-w^2]);
Ad: mat_function(exp, A*T);
Bd: expand(integrate(mat_function(exp, A*t), t, 0, T).B);
D: matrix([1, 0, -1/w^2]);

Anew: (Ad - Bd.D/T);
Bnew: Bd/T;
Dnew: -D/T;
Enew: matrix([1/T, 0], [0, 1/T]);

Anew: hypsimp(w*T, Anew);
Bnew: hypsimp(w*T, Bnew);

tex(Anew);
tex(Bnew);
\end{deflisting}
\end{listingtcb}

\let\cpos\THISCOMMANDSHOULDNEVERBEDEFINED
\let\cvel\THISCOMMANDSHOULDNEVERBEDEFINED
\let\cacc\THISCOMMANDSHOULDNEVERBEDEFINED
\let\cjerk\THISCOMMANDSHOULDNEVERBEDEFINED

\let\cPos\THISCOMMANDSHOULDNEVERBEDEFINED
\let\cVel\THISCOMMANDSHOULDNEVERBEDEFINED
\let\cAcc\THISCOMMANDSHOULDNEVERBEDEFINED
\let\cJerk\THISCOMMANDSHOULDNEVERBEDEFINED

\let\cstate\THISCOMMANDSHOULDNEVERBEDEFINED
\let\cState\THISCOMMANDSHOULDNEVERBEDEFINED

\let\cop\THISCOMMANDSHOULDNEVERBEDEFINED
\let\dcop\THISCOMMANDSHOULDNEVERBEDEFINED
\let\CoP\THISCOMMANDSHOULDNEVERBEDEFINED
\let\dCoP\THISCOMMANDSHOULDNEVERBEDEFINED

\let\fph\THISCOMMANDSHOULDNEVERBEDEFINED
\let\FPh\THISCOMMANDSHOULDNEVERBEDEFINED

\let\fp\THISCOMMANDSHOULDNEVERBEDEFINED
\let\FP\THISCOMMANDSHOULDNEVERBEDEFINED

\let\fd\THISCOMMANDSHOULDNEVERBEDEFINED
\let\FD\THISCOMMANDSHOULDNEVERBEDEFINED

\let\cp\THISCOMMANDSHOULDNEVERBEDEFINED



\subimport*{../../modules/wpg04/doc/}{wpg04.tex}

\printbibliography[heading=bibintoc, title=Bibliography]
\end{refsection}
